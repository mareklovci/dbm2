% \linespread{1}

\newcommand{\modelmatrix}[1]{\begin{bmatrix} {#1}_{k, 1}\\ {#1}_{k, 2} \end{bmatrix}}
\newcommand{\myvector}[2]{\begin{bmatrix} {#1}\\ {#2} \end{bmatrix}}

\section{Zadání}

Uvažujte problém detekce aditivních chyb senzorů pro měření rychlosti průtoku ideální nestlačitelné kapaliny v potrubí. Rychlost proudění je měřena dvěma senzory rychlosti proudění ve dvou místech potrubí, která jsou od sebe vzdálena \( 1 \: m \) a mají různé průřezy \( S_1 = 0.03 \: m^2 \) a \( S_2 = 0.02 \: m^2 \). Předpokládejte, že takový systém lze popsat diskrétním modelem 

\begin{align}
    \modelmatrix{y} =
    \begin{bmatrix} \frac{1}{S_1}\\ \frac{1}{S_2} \end{bmatrix}
    x_k + \modelmatrix{f} + \modelmatrix{v}
\end{align}

kde \( v_{k, i} \: m \cdot s^{-1} \) je vvýstup senzoru rychlosti průtoku \( i \), \( x_k \: m^3 \cdot s^{-1} \) je neznámý průtok, \( f_{k, i} \) reprezentuje aditivní chybu senzoru \( i \) a \( v_{k, i} \) je šum měření senzoru \( i \). Předpokládejte, že \( v_{k, 1} \) je bílý šum s normálním rozdělením pravděpodobnosti s nulovou střední hodnotou a variancí \( \sigma^2_1 = 1 \cdot 10^{−4} \). Šum měření druhého senzoru je generován dle předpisu

\begin{align}
    v_{k, 2} = 0.2 v_{k, 1} + \xi_k
\end{align}

kde \( \xi_k \) je bílý šum s normálním rozdělením pravděpodobnosti s nulovou střední hodnotou a variancí \( \sigma^2_\xi = 4 \cdot 10^{−4} \). Náhodné procesy \( v_{k, 1} \) a \( \xi_k \) jsou vzájemně nezávislé.

\begin{enumerate}
    \item Návrh generátoru reziduí
    
    \begin{itemize}
        \item Navrhněte generátor reziduí tak, aby reziduální signál \( r_k \) nezávisel na neznámém průtoku \( x_k \).
        
        \item Vypočtěte střední hodnotu a varianci reziduálního signálu \( r_k \) generovaného navrženým generátorem. Diskutujte, jak závisí tyto momenty náhodné veličiny na aditivních chybách senzorů \( f_{k, 1} \) a \( f_{k, 2} \).
    \end{itemize}
    
    \item Návrh generátoru rozhodnutí
    
    \begin{itemize}
        \item Předpokládejte, že v neznámém časovém okamžiku \( k_f \) vznikne v prvním senzoru aditivní chyba o velikosti 0.01, tj.
        
        \begin{align}
            \modelmatrix{f} = \myvector{0}{0} \text{pro} \: k < k_f \text{,} \myvector{0.01}{0} \text{pro} \: k \geq k_f
        \end{align}
        
        \item Navrhněte generátor rozhodnutí s využitím testu kumulativní sumy. Jako hodnoty pro mez testu \( S_{mez} \) volte postupně \( \{1, 2, 3,\ldots, 20\} \) a pro každou hodnotu odhadněte pravděpodobnosti falešného alarmu \( \alpha_0 \) a nedetekování chyby \( \alpha_1 \) pomocí 1000 Monte Carlo simulací. Za nedetekování chyby považujte situaci, kdy je chyba detekována po více jak 40 časových krocích po jejím vzniku.
    \end{itemize}
    
    \item Monte Carlo simulace
    
    \begin{itemize}
        \item Pro provedení Monte Carlo simulací použijte jako neznámý průtok signál 
        
        \begin{align}
            x_k = 0.01 \sin{(0.01 k + 0.2)}
        \end{align}
        
        časový okamžik vzniku chyby \( k_f = 250 \).
        
        \item Vykreslete časové průběhy průtoku, výstupů senzorů, chyb, reziduálního signálu, statistiky testu kumulativní sumy a výsledné rozhodnutí pro jednu zvolenou hodnotu \( S_{mez} \) a jednu Monte Carlo realizaci. 
        
        \item Diskutujte závislost odhadů pravděpodobností falešných alarmů a nedetekování chyby na velikosti meze \( S_{mez} \).
    \end{itemize}
\end{enumerate}
